\documentclass[]{book}
\usepackage{lmodern}
\usepackage{amssymb,amsmath}
\usepackage{ifxetex,ifluatex}
\usepackage{fixltx2e} % provides \textsubscript
\ifnum 0\ifxetex 1\fi\ifluatex 1\fi=0 % if pdftex
  \usepackage[T1]{fontenc}
  \usepackage[utf8]{inputenc}
\else % if luatex or xelatex
  \ifxetex
    \usepackage{mathspec}
  \else
    \usepackage{fontspec}
  \fi
  \defaultfontfeatures{Ligatures=TeX,Scale=MatchLowercase}
\fi
% use upquote if available, for straight quotes in verbatim environments
\IfFileExists{upquote.sty}{\usepackage{upquote}}{}
% use microtype if available
\IfFileExists{microtype.sty}{%
\usepackage[]{microtype}
\UseMicrotypeSet[protrusion]{basicmath} % disable protrusion for tt fonts
}{}
\PassOptionsToPackage{hyphens}{url} % url is loaded by hyperref
\usepackage[unicode=true]{hyperref}
\hypersetup{
            pdftitle={Datenanalyse: Eye-tracking 2020/2021},
            pdfauthor={Katja Suckow},
            pdfborder={0 0 0},
            breaklinks=true}
\urlstyle{same}  % don't use monospace font for urls
\usepackage{natbib}
\bibliographystyle{apalike}
\usepackage{color}
\usepackage{fancyvrb}
\newcommand{\VerbBar}{|}
\newcommand{\VERB}{\Verb[commandchars=\\\{\}]}
\DefineVerbatimEnvironment{Highlighting}{Verbatim}{commandchars=\\\{\}}
% Add ',fontsize=\small' for more characters per line
\usepackage{framed}
\definecolor{shadecolor}{RGB}{248,248,248}
\newenvironment{Shaded}{\begin{snugshade}}{\end{snugshade}}
\newcommand{\KeywordTok}[1]{\textcolor[rgb]{0.13,0.29,0.53}{\textbf{#1}}}
\newcommand{\DataTypeTok}[1]{\textcolor[rgb]{0.13,0.29,0.53}{#1}}
\newcommand{\DecValTok}[1]{\textcolor[rgb]{0.00,0.00,0.81}{#1}}
\newcommand{\BaseNTok}[1]{\textcolor[rgb]{0.00,0.00,0.81}{#1}}
\newcommand{\FloatTok}[1]{\textcolor[rgb]{0.00,0.00,0.81}{#1}}
\newcommand{\ConstantTok}[1]{\textcolor[rgb]{0.00,0.00,0.00}{#1}}
\newcommand{\CharTok}[1]{\textcolor[rgb]{0.31,0.60,0.02}{#1}}
\newcommand{\SpecialCharTok}[1]{\textcolor[rgb]{0.00,0.00,0.00}{#1}}
\newcommand{\StringTok}[1]{\textcolor[rgb]{0.31,0.60,0.02}{#1}}
\newcommand{\VerbatimStringTok}[1]{\textcolor[rgb]{0.31,0.60,0.02}{#1}}
\newcommand{\SpecialStringTok}[1]{\textcolor[rgb]{0.31,0.60,0.02}{#1}}
\newcommand{\ImportTok}[1]{#1}
\newcommand{\CommentTok}[1]{\textcolor[rgb]{0.56,0.35,0.01}{\textit{#1}}}
\newcommand{\DocumentationTok}[1]{\textcolor[rgb]{0.56,0.35,0.01}{\textbf{\textit{#1}}}}
\newcommand{\AnnotationTok}[1]{\textcolor[rgb]{0.56,0.35,0.01}{\textbf{\textit{#1}}}}
\newcommand{\CommentVarTok}[1]{\textcolor[rgb]{0.56,0.35,0.01}{\textbf{\textit{#1}}}}
\newcommand{\OtherTok}[1]{\textcolor[rgb]{0.56,0.35,0.01}{#1}}
\newcommand{\FunctionTok}[1]{\textcolor[rgb]{0.00,0.00,0.00}{#1}}
\newcommand{\VariableTok}[1]{\textcolor[rgb]{0.00,0.00,0.00}{#1}}
\newcommand{\ControlFlowTok}[1]{\textcolor[rgb]{0.13,0.29,0.53}{\textbf{#1}}}
\newcommand{\OperatorTok}[1]{\textcolor[rgb]{0.81,0.36,0.00}{\textbf{#1}}}
\newcommand{\BuiltInTok}[1]{#1}
\newcommand{\ExtensionTok}[1]{#1}
\newcommand{\PreprocessorTok}[1]{\textcolor[rgb]{0.56,0.35,0.01}{\textit{#1}}}
\newcommand{\AttributeTok}[1]{\textcolor[rgb]{0.77,0.63,0.00}{#1}}
\newcommand{\RegionMarkerTok}[1]{#1}
\newcommand{\InformationTok}[1]{\textcolor[rgb]{0.56,0.35,0.01}{\textbf{\textit{#1}}}}
\newcommand{\WarningTok}[1]{\textcolor[rgb]{0.56,0.35,0.01}{\textbf{\textit{#1}}}}
\newcommand{\AlertTok}[1]{\textcolor[rgb]{0.94,0.16,0.16}{#1}}
\newcommand{\ErrorTok}[1]{\textcolor[rgb]{0.64,0.00,0.00}{\textbf{#1}}}
\newcommand{\NormalTok}[1]{#1}
\usepackage{longtable,booktabs}
% Fix footnotes in tables (requires footnote package)
\IfFileExists{footnote.sty}{\usepackage{footnote}\makesavenoteenv{long table}}{}
\usepackage{graphicx,grffile}
\makeatletter
\def\maxwidth{\ifdim\Gin@nat@width>\linewidth\linewidth\else\Gin@nat@width\fi}
\def\maxheight{\ifdim\Gin@nat@height>\textheight\textheight\else\Gin@nat@height\fi}
\makeatother
% Scale images if necessary, so that they will not overflow the page
% margins by default, and it is still possible to overwrite the defaults
% using explicit options in \includegraphics[width, height, ...]{}
\setkeys{Gin}{width=\maxwidth,height=\maxheight,keepaspectratio}
\IfFileExists{parskip.sty}{%
\usepackage{parskip}
}{% else
\setlength{\parindent}{0pt}
\setlength{\parskip}{6pt plus 2pt minus 1pt}
}
\setlength{\emergencystretch}{3em}  % prevent overfull lines
\providecommand{\tightlist}{%
  \setlength{\itemsep}{0pt}\setlength{\parskip}{0pt}}
\setcounter{secnumdepth}{5}
% Redefines (sub)paragraphs to behave more like sections
\ifx\paragraph\undefined\else
\let\oldparagraph\paragraph
\renewcommand{\paragraph}[1]{\oldparagraph{#1}\mbox{}}
\fi
\ifx\subparagraph\undefined\else
\let\oldsubparagraph\subparagraph
\renewcommand{\subparagraph}[1]{\oldsubparagraph{#1}\mbox{}}
\fi

% set default figure placement to htbp
\makeatletter
\def\fps@figure{htbp}
\makeatother

\usepackage{booktabs}

\title{Datenanalyse: Eye-tracking 2020/2021}
\author{Katja Suckow}
\date{2021-01-07}

\begin{document}
\maketitle

{
\setcounter{tocdepth}{1}
\tableofcontents
}
\chapter{Willkommen}\label{willkommen}

Da wir uns leider in diesem Jahr erstmal nicht persönlich treffen
können, habe ich hier eine kleine Übersicht zu den verwendeten
R-Befehlen erstellt.

\section{Was braucht man dafür?}\label{was-braucht-man-dafuxfcr}

Dafür muss man auf seinem Rechner \textbf{R} und \textbf{Rstudio}
installieren (beide sollten auf Windows, Mac und Linux laufen).

\begin{enumerate}
\def\labelenumi{\arabic{enumi}.}
\tightlist
\item
  \textbf{R} frei verfügbare Software \url{https://www.r-project.org/}
\end{enumerate}

\includegraphics{./img/Rdownload.png} \textbf{R} ist eine freie
Programmiersprache zur statistischen Datenanalyse und Erstellung von
Grafiken. Sie kann auch als Skriptsprache benutzt werden, um einfache
Skripte und Programme zu schreiben. \textbf{R} ist vor allem in der
Wissenschaft weit verbreitet und löst hier zunehmend SPSS ab. \textbf{R}
bietet viele Methoden und Pakete zu statistischen Auswertung und
Datendarstellung, die ständig in open-source weiterentwickelt werden, es
steht kostenlos zur Verfügung und kann auf jeder Plattform (Windows,
Mac, Linux) laufen. R besteht aus 3 Hauptfenstern: (1) \textbf{Konsole},
um direkt Befehle einzugeben, (2) \textbf{Editor}, um eine Abfolge an
Befehlen zu speichern und auszuführen, (3) \textbf{Grafikfenster}

\begin{enumerate}
\def\labelenumi{\arabic{enumi}.}
\setcounter{enumi}{1}
\tightlist
\item
  \textbf{RStudio}
  \url{https://www.rstudio.com/products/rstudio/download} Es gibt
  verschiedene Möglichkeiten mit R zu arbeiten. Wir werden die grafische
  Oberfläche, die \textbf{RStudio} bietet, nutzen. \textbf{RStudio}
  bietet eine gut handhabe Oberfläche, viel Unterstützung und viele
  integrierte Apps (Shiny, Markdown, Bookdown \ldots{}), die auf R
  zurückgreifen. Unterschiedliche Editoren zur Bearbeitung von R Dateien
  sind:
\end{enumerate}

\begin{itemize}
\tightlist
\item
  RStudio (für die gemeinsame Datenauswertung im Praktikum empfohlen)
\item
  Notepad++ (Windows)
\item
  Textwrangler
\item
  \(\dots\)
\end{itemize}

\begin{figure}
\centering
\includegraphics{./img/download.jpg}
\caption{RStudio Logo}
\end{figure}

\includegraphics{./img/RStudio-look1.png} So sieht RStudio ohne Inhalt
oder Daten aus.

\begin{figure}
\centering
\includegraphics{./img/RStudio-look.png}
\caption{Die 4 Hauptfenster von RStudio}
\end{figure}

In der Standarddarstellung befindet sich oben links der \textbf{Code
Editor}, oben rechts können sie \textbf{Workspace und History} ansehen,
unten links befindet sich die \textbf{R Console}, in der Sie den Code
direkt eingeben können und unten rechts lassen sich unter anderem
\textbf{Plots und Files} anzeigen. Schauen Sie sich die verschiedenen
Möglichkeiten an, die die verschiedenen Tabs in den Ecken bieten.

\textbf{Skript}

Das Skript enthält eine Sammlung von Befehlen. Diese können auch direkt
über Konsole ausgeführt werden. Es empfiehlt sich jedoch sehr, diese im
\textbf{Code Editor} zu speichern, um später noch einmal darauf
zugreifen zu können.

\textbf{Workspace}

Der Workspace enthält Sammlung von Objekten, die in einer Session
erstellt wurden. Diese kann explizit gespeichert werden, um die
einzelnen Objekte noch einmal neu zu laden.

So sieht Rstudio mit ein bischen mehr Inhalt aus.

\begin{figure}
\centering
\includegraphics{./img/ScreenGrafik.png}
\caption{RStudio mit Datensatz und grafischer Darstellung eines
Datensatzes}
\end{figure}

\begin{itemize}
\tightlist
\item
  \textbf{oben links} steht der Code, der gespeichert und auch aus dem
  Editor ausgeführt werden kann. (Mac: Markierung des auszuführenden
  Codes und dann \textbf{CMD + Enter})
\item
  \textbf{unten links} man kannn den auszuführenden Code auch direkt
  unten in das Konsolenfenster eingeben. Dann wird dieser allerdings
  nicht gespeichert. Es wird empfohlen, Code der erstmal nur ausprobiert
  wird, direkt in das Konsolenfenster einzutragen. Wenn der Code das
  korrekte Ergebnis liefert, sollte er im Editor in einem Skript
  gespeichert werden.
\item
  \textbf{oben rechts} lassen sich History der eingegebenen Befehle und
  auch die Inhalte des Workspace anzeigen
\item
  \textbf{unten rechts} kann man sich die geladenen Pakete für
  \textbf{R}, sowie die verschiedenen erstellten Grafiken anzeigen
  lassen
\end{itemize}

Die Anordnung der Fenster lassen sich über \textbf{RStudio}
\textgreater{} \textbf{Preferences} und dann \textbf{Pane Layout}
ändern. Über \textbf{Preferences} können Sie sich generell anzeigen
lassen, welche Änderungen Sie u.a. bei der Anzeige vornehmen können
bzw.\textasciitilde{}wollen.

\begin{enumerate}
\def\labelenumi{\arabic{enumi}.}
\setcounter{enumi}{2}
\tightlist
\item
  \textbf{Shiny App}
\end{enumerate}

Das ist eine App, die in RStudio integriert ist und die deskriptive
Daten mit \textbf{R} bildlich und interaktiv veranschaulichen kann und
auch veröffentlicht. Wenn man sich einmal ein bisschen mit R
auseinandergesetzt hat, kann man sich mit Shiny weiter ``austoben''.

\begin{figure}
\centering
\includegraphics{./img/Shiny-Example.png}
\caption{Beispiel des Screenshots einer Shiny App, die interaktiv die
Normalverteilung einer übergebenen Anzahl beobachteter Werte anzeigt.}
\end{figure}

\section{Einleitung: Datentypen in der
Linguistik}\label{einleitung-datentypen-in-der-linguistik}

\textbf{Qualitative Daten} Nicht-numerische, oft verschriftlichte oder
in audiovisueller Form vorliegende Daten. Verwendung häufig explorativ
und hypothesengenerierend. (e.g., Interviews)

\textbf{Quantitative Daten} Numerische (bzw. in numerischer Form
überführte) Daten, die mit dem Ziel der Überprüfung von Hypothesen und
Theorien erhoben werden

\section{Population und Stichprobe}\label{population-und-stichprobe}

In der psycholinguistischen Forschung werden Daten in einer
kontrollierten Umgebung (Experiment) erhoben, die Aussagen für die
Gesamtpopulation zu einer bestimmten linguistischen Fragestellung machen
zu können.

Dabei wird eine kleine Gruppe der Population (Stichprobe) in einem
Experiment getestet. Bestimmte Daten dieser Stichprobe werden erhoben
und anschliessend die Ergebnisse dieser Daten auf die Gesamtpopulation
projiziert. Die erhobenen Daten der Stichprobe müssen erst in
deskriptive statistische Kennwerte umgewandelt werden und anschliessend
mittels statistischer Verfahren Hypothesen geprüft werden, um Aussagen
über die Gesamtpopulation machen zu können.

\section{Deskriptive Statistik und
Inferenzstatistik}\label{deskriptive-statistik-und-inferenzstatistik}

\textbf{Deskriptive Statistik} übersichtliche Darstellung der erhobenen
Daten einer Stichprobe in Form statistischer Kennwerte (Mittelwerte,
Streuung, Verteilung, Grafiken)

\textbf{Inferenzstatistik} Statistische Verfahren, die es ermöglichen
Rückschlüsse aus den Daten auf die Gesamtpopulation zu ziehen, um
Schätzungen für Populationswerte zu berechnen oder experimentelle
Hypothesen zu testen.

\section{R als Taschenrechner:
Beispiele}\label{r-als-taschenrechner-beispiele}

Um sich erstmal mit \textbf{R} vertraut zu machen, kann man R erstmal
als Taschenrechner einsetzen. Dabei ist die Anwendung recht assoziativ
und kann wie folgt eingesetzt werden.

Probieren Sie diese Beispiele erstmal selbst zuerst in Ihrem Terminal
und dann aus dem Editor heraus aus!

\textbf{Addition + }

\begin{Shaded}
\begin{Highlighting}[]
\DecValTok{5} \OperatorTok{+}\StringTok{ }\DecValTok{7}
\end{Highlighting}
\end{Shaded}

\begin{verbatim}
## [1] 12
\end{verbatim}

\textbf{Subtraktion - }

\begin{Shaded}
\begin{Highlighting}[]
\DecValTok{325} \OperatorTok{-}\StringTok{ }\DecValTok{18}
\end{Highlighting}
\end{Shaded}

\begin{verbatim}
## [1] 307
\end{verbatim}

\textbf{Multiplikation * }

\begin{Shaded}
\begin{Highlighting}[]
\DecValTok{43} \OperatorTok{*}\StringTok{ }\DecValTok{21}
\end{Highlighting}
\end{Shaded}

\begin{verbatim}
## [1] 903
\end{verbatim}

\textbf{Division / }

\begin{Shaded}
\begin{Highlighting}[]
\DecValTok{442} \OperatorTok{/}\StringTok{ }\DecValTok{13}
\end{Highlighting}
\end{Shaded}

\begin{verbatim}
## [1] 34
\end{verbatim}

\textbf{Exponent \^{} }

\begin{Shaded}
\begin{Highlighting}[]
\DecValTok{32}\OperatorTok{^}\DecValTok{2}
\end{Highlighting}
\end{Shaded}

\begin{verbatim}
## [1] 1024
\end{verbatim}

\section{Einige vordefinierte
Beispielfunktionen}\label{einige-vordefinierte-beispielfunktionen}

Funktionen können auch sehr einfach in \textbf{R} selbst geschrieben
werden. Dazu kommenw wir später auch noch. Man muss allerdings nicht
alles neu erfinden, besonders weil \textbf{R} auch viele Funktionen
vordefiniert mitbringt.

Die Syntax ist relativ simpel. Die Funktion hat einen eigenen Namen und
die übergebenen Argumente, mit denen die Funktion Berechnungen
durchführt werden in der Klammer nach dem Funktionsnamen übergeben.

Unten sind ein paar Beispiele für frequente leicht nachvollziehbare
Funktionen aufgelistet.

\textbf{Summe sum()} - Bildung der Summe, der übergebenen Argumente:

\begin{Shaded}
\begin{Highlighting}[]
\KeywordTok{sum}\NormalTok{(}\DecValTok{3}\NormalTok{,}\DecValTok{45}\NormalTok{,}\DecValTok{12}\NormalTok{,}\DecValTok{34}\NormalTok{)}
\end{Highlighting}
\end{Shaded}

\begin{verbatim}
## [1] 94
\end{verbatim}

\textbf{Wurzel sqrt()} - Bildung der Wurzel von einem übergebenen
Argument:

\begin{Shaded}
\begin{Highlighting}[]
\KeywordTok{sqrt}\NormalTok{(}\DecValTok{1024}\NormalTok{)}
\end{Highlighting}
\end{Shaded}

\begin{verbatim}
## [1] 32
\end{verbatim}

\textbf{Minimum min()} - Ermittlung der kleinsten Zahl aus einer Liste
von übergebenen Werten:

\begin{Shaded}
\begin{Highlighting}[]
\KeywordTok{min}\NormalTok{(}\DecValTok{43}\NormalTok{,}\DecValTok{24}\NormalTok{,}\DecValTok{11}\NormalTok{,}\DecValTok{23}\NormalTok{,}\DecValTok{76}\NormalTok{,}\DecValTok{14}\NormalTok{,}\DecValTok{56}\NormalTok{,}\DecValTok{99}\NormalTok{,}\DecValTok{12}\NormalTok{)}
\end{Highlighting}
\end{Shaded}

\begin{verbatim}
## [1] 11
\end{verbatim}

\textbf{Maximum max()} - Ermittlung der höchsten Zahl aus einer Liste
von übergebenen Werten:

\begin{Shaded}
\begin{Highlighting}[]
\KeywordTok{max}\NormalTok{(}\DecValTok{43}\NormalTok{,}\DecValTok{24}\NormalTok{,}\DecValTok{11}\NormalTok{,}\DecValTok{23}\NormalTok{,}\DecValTok{76}\NormalTok{,}\DecValTok{14}\NormalTok{,}\DecValTok{56}\NormalTok{,}\DecValTok{99}\NormalTok{,}\DecValTok{12}\NormalTok{)}
\end{Highlighting}
\end{Shaded}

\begin{verbatim}
## [1] 99
\end{verbatim}

\textbf{Absoluter Wert (Betrag) abs()} - Ermittlung des absoluten
Betrags aus einem Wert:

\begin{Shaded}
\begin{Highlighting}[]
\KeywordTok{abs}\NormalTok{(}\OperatorTok{-}\DecValTok{25}\NormalTok{)}
\end{Highlighting}
\end{Shaded}

\begin{verbatim}
## [1] 25
\end{verbatim}

\textbf{Aneinanderreihung concatenate c()} verbindet in Klammern
stehende Zahlen zu einer Liste

\begin{Shaded}
\begin{Highlighting}[]
\KeywordTok{c}\NormalTok{(}\DecValTok{2}\NormalTok{,}\DecValTok{6}\NormalTok{,}\DecValTok{7}\NormalTok{)}
\end{Highlighting}
\end{Shaded}

\textbf{List ls()}

\begin{itemize}
\tightlist
\item
  listet alle Elemente im aktuellen Workspace auf
\end{itemize}

\begin{Shaded}
\begin{Highlighting}[]
\KeywordTok{ls}\NormalTok{()}
\end{Highlighting}
\end{Shaded}

\textbf{Gib Pfad des aktuellen working directory zurück getwd()}

\begin{itemize}
\tightlist
\item
  Gibt Verzeichnis des aktuellen Workspace aus
\end{itemize}

\begin{Shaded}
\begin{Highlighting}[]
\KeywordTok{getwd}\NormalTok{()}
\end{Highlighting}
\end{Shaded}

\textbf{Setze working directory auf folgenden Pfad setwd()}

\begin{itemize}
\tightlist
\item
  Setzt den aktuellen Workspace in angegebenes Verzeichnis
\end{itemize}

\begin{Shaded}
\begin{Highlighting}[]
\KeywordTok{setwd}\NormalTok{(}\OperatorTok{<}\NormalTok{Pfad}\OperatorTok{>}\NormalTok{)}
\end{Highlighting}
\end{Shaded}

\textbf{Lies Datensatz ein read.table()}

\begin{itemize}
\tightlist
\item
  liest einen Datensatz in aktuellen Workspace ein
\end{itemize}

\begin{Shaded}
\begin{Highlighting}[]
\KeywordTok{read.table}\NormalTok{(}\OperatorTok{<}\NormalTok{file}\OperatorTok{>}\NormalTok{, }\DataTypeTok{header =}\NormalTok{ T)}
\end{Highlighting}
\end{Shaded}

\textbf{write.table()}

\begin{itemize}
\tightlist
\item
  exportiert eine Variable aus aktuellem working directory in die
  angegebene Dateinamen
\end{itemize}

\begin{Shaded}
\begin{Highlighting}[]
\KeywordTok{write.table}\NormalTok{(}\OperatorTok{<}\NormalTok{Variablenname}\OperatorTok{>}\NormalTok{,}\DataTypeTok{file =} \StringTok{""}\NormalTok{)}
\end{Highlighting}
\end{Shaded}

\section{Vergleichsoperatoren}\label{vergleichsoperatoren}

Hier sind die Beispiele logischer Vergleichsoperatoren. Deren
Interpretation gibt entweder \textbf{TRUE} oder \textbf{FALSE} zurück.

gleich \textbf{==}

\begin{itemize}
\tightlist
\item
  Vergleich ob die Werte beider Ausdrücke identisch sind
\item
  ja: \textbf{TRUE}
\item
  nein: \textbf{FALSE}
\end{itemize}

\begin{Shaded}
\begin{Highlighting}[]
\DecValTok{4} \OperatorTok{==}\StringTok{ }\DecValTok{6}
\end{Highlighting}
\end{Shaded}

\begin{verbatim}
## [1] FALSE
\end{verbatim}

ungleich: \textbf{!=}

\begin{itemize}
\tightlist
\item
  Vergleich ob die Werte beider Ausdrücke ungleich sind
\item
  ja sie sind ungleich: \textbf{TRUE}
\item
  nein sie sind gleich: \textbf{FALSE}
\end{itemize}

\begin{Shaded}
\begin{Highlighting}[]
\DecValTok{4} \OperatorTok{!=}\StringTok{ }\DecValTok{6}
\end{Highlighting}
\end{Shaded}

\begin{verbatim}
## [1] TRUE
\end{verbatim}

größer: \textbf{\textgreater{}}

\begin{itemize}
\tightlist
\item
  ist der Wert links größer als der Wert rechts?
\end{itemize}

\begin{Shaded}
\begin{Highlighting}[]
\DecValTok{4} \OperatorTok{>}\StringTok{ }\DecValTok{6}
\end{Highlighting}
\end{Shaded}

\begin{verbatim}
## [1] FALSE
\end{verbatim}

kleiner: \textbf{\textless{}}

\begin{itemize}
\tightlist
\item
  ist der Wert links kleiner als der Wert rechts?
\end{itemize}

\begin{Shaded}
\begin{Highlighting}[]
\DecValTok{4} \OperatorTok{<}\StringTok{ }\DecValTok{6}
\end{Highlighting}
\end{Shaded}

\begin{verbatim}
## [1] TRUE
\end{verbatim}

größer gleich: \textbf{\textgreater{}=}

\begin{itemize}
\tightlist
\item
  ist der Wert links größer oder gleich gleich der Wert rechts?
\end{itemize}

\begin{Shaded}
\begin{Highlighting}[]
\DecValTok{4} \OperatorTok{>=}\StringTok{ }\DecValTok{6}
\end{Highlighting}
\end{Shaded}

\begin{verbatim}
## [1] FALSE
\end{verbatim}

\begin{Shaded}
\begin{Highlighting}[]
\DecValTok{4} \OperatorTok{>=}\StringTok{ }\DecValTok{4}
\end{Highlighting}
\end{Shaded}

\begin{verbatim}
## [1] TRUE
\end{verbatim}

kleiner gleich: \textbf{\textless{}=}

\begin{itemize}
\tightlist
\item
  ist der Wert links kleiner als der Wert rechts?
\end{itemize}

\begin{Shaded}
\begin{Highlighting}[]
\DecValTok{4} \OperatorTok{<=}\StringTok{ }\DecValTok{6}
\end{Highlighting}
\end{Shaded}

\begin{verbatim}
## [1] TRUE
\end{verbatim}

\begin{Shaded}
\begin{Highlighting}[]
\DecValTok{4} \OperatorTok{<=}\StringTok{ }\DecValTok{4}
\end{Highlighting}
\end{Shaded}

\begin{verbatim}
## [1] TRUE
\end{verbatim}

\section{Logische
Vergleichsoperatoren}\label{logische-vergleichsoperatoren}

Konjunktion, logisches ``Und'': \textbf{\&\&} oder \textbf{\&}

\begin{itemize}
\tightlist
\item
  beim logischen \textbf{Und} werden die Werte der beiden Ausdrücke
  verglichen - nur wenn beide \textbf{TRUE} ergeben, wird der
  Gesamtausdruck \textbf{TRUE} ergeben
\item
  im unteren Beispiele sind beide Ausdrück \textbf{TRUE}
\end{itemize}

\begin{Shaded}
\begin{Highlighting}[]
\DecValTok{4} \OperatorTok{<=}\StringTok{ }\DecValTok{4} \OperatorTok{&}\StringTok{ }\DecValTok{4} \OperatorTok{>=}\StringTok{ }\DecValTok{4}
\end{Highlighting}
\end{Shaded}

\begin{verbatim}
## [1] TRUE
\end{verbatim}

Disjunktion, logisches ``oder'': \textbf{\textbar{}\textbar{}} oder
\textbf{\textbar{}}

\begin{itemize}
\tightlist
\item
  beim logischen \textbf{Oder} reicht es, wenn ein zu vergleichender
  Ausdruck \textbf{TRUE} ist, um den gesamten Ausdruck \textbf{TRUE} zu
  machen.
\end{itemize}

\begin{Shaded}
\begin{Highlighting}[]
\DecValTok{4} \OperatorTok{<}\StringTok{ }\DecValTok{6} \OperatorTok{||}\StringTok{ }\DecValTok{5} \OperatorTok{>}\StringTok{ }\DecValTok{8}
\end{Highlighting}
\end{Shaded}

\begin{verbatim}
## [1] TRUE
\end{verbatim}

Negation: \textbf{!}

\begin{itemize}
\tightlist
\item
  bei der Negation wird der Wahrheitsgehalt eines Ausdruckes umgekehrt.
\end{itemize}

\begin{Shaded}
\begin{Highlighting}[]
\OperatorTok{!}\NormalTok{(}\DecValTok{4} \OperatorTok{<}\StringTok{ }\DecValTok{6} \OperatorTok{||}\StringTok{ }\DecValTok{5} \OperatorTok{>}\StringTok{ }\DecValTok{8}\NormalTok{)}
\end{Highlighting}
\end{Shaded}

\begin{verbatim}
## [1] FALSE
\end{verbatim}

\section{Variablen in R}\label{variablen-in-r}

Verschiedene Werte oder Ergebnisse einer Berechnung können leicht in
Variablen (Platzhalter) gespeichert werden. Diese Variablen müssen in
\textbf{R} mit einem Buchstaben beginnen. Die Zuweisung von Werten in
eine Variable erfolgt entweder mit einem Pfeil \textbf{\textless{}-}
oder mit einem simplen \textbf{=} (beides ohne Unterschied möglich):

\begin{Shaded}
\begin{Highlighting}[]
\NormalTok{a <-}\StringTok{ }\DecValTok{15}
\NormalTok{b =}\StringTok{ }\DecValTok{23}
\NormalTok{ab =}\StringTok{ }\KeywordTok{c}\NormalTok{(a,b)}
\NormalTok{ab}
\end{Highlighting}
\end{Shaded}

\begin{verbatim}
## [1] 15 23
\end{verbatim}

\section{Hilfefunktionen}\label{hilfefunktionen}

Mit der Hilfefunktion \textbf{?} lässt sich zu jeder R-Funktion eine
hilfreiche Beschreibung anzeigen. Schreiben Sie einfach das \textbf{?}
direkt vor die Funktion zu der Sie mehr erfahren wollen und \textbf{R}
öffnet ein Hilfsfenster. Probieren Sie es einmal aus und schreiben:

\textbf{?sum}

Mit \textbf{?sum} können Sie sich die Informationen zum \textbf{sum}
Befehl anzeigen lassen. Meist werden am Ende der Beschreibung auch
Beispiele zur Benutzung angezeigt.

Alternativ können Sie auch den \textbf{help()} benutzen:

\begin{Shaded}
\begin{Highlighting}[]
\NormalTok{?sum}
\KeywordTok{help}\NormalTok{(sum)}
\end{Highlighting}
\end{Shaded}

\section{Pakete Laden}\label{pakete-laden}

Pakete sind eine Sammlung verschiedener Funktionen zu einem Thema.

\begin{Shaded}
\begin{Highlighting}[]
\CommentTok{# installiert das Paket}
\KeywordTok{install.packages}\NormalTok{(}\StringTok{"tidyverse"}\NormalTok{)}

\CommentTok{# lädt das Paket}
\KeywordTok{library}\NormalTok{(tidyverse)}
\end{Highlighting}
\end{Shaded}

\chapter{Datenbeschreibungen in R}\label{datenbeschreibungen-in-r}

Kurze Zusammenfassung einiger zentraler Funktionen für das Praktikum.

\begin{Shaded}
\begin{Highlighting}[]
\CommentTok{# Daten einlesen}
\KeywordTok{getwd}\NormalTok{()}
\CommentTok{# Der Pfad zu Eurem Workspace ist auf jedem Rechner anders}
\CommentTok{# Deshalb läßt sich das nicht vorgeben und Ihr könnt diesen nicht einfach}
\CommentTok{# von jemand kopieren.}
\KeywordTok{setwd}\NormalTok{(}\StringTok{"<Setzt bitte hier den Pfad zum Workspace auf Eurem Rechner ein>"}\NormalTok{)}
\CommentTok{# Der Datensatz sollte 10VP.dat sollte im Pfad Eures Workspace gespeichert sein}
\NormalTok{list =}\StringTok{ }\KeywordTok{read.table}\NormalTok{(}\StringTok{"gugVP11.txt"}\NormalTok{,}\DataTypeTok{header =}\NormalTok{ T)}
\end{Highlighting}
\end{Shaded}

\begin{Shaded}
\begin{Highlighting}[]
\CommentTok{# Inhalt des Workspace}
\CommentTok{# welche Variablen wurden eingelesen oder selbst erstellt}
\KeywordTok{ls}\NormalTok{()}

\CommentTok{# listet die ersten 6 Zeilen auf}
\KeywordTok{head}\NormalTok{(list)}

\CommentTok{# Berechne die Summe aller Lesezeiten}
\KeywordTok{sum}\NormalTok{(list}\OperatorTok{$}\NormalTok{time)}
\end{Highlighting}
\end{Shaded}

\section{R-Vektoren}\label{r-vektoren}

Vektoren sind eine Datenstruktur, die Elemente desselben Datentyps in
Form einer Liste enhält. Diese Datentypen können \emph{logisch},
\emph{integer}, \emph{double}, \emph{character} oder \emph{komplex}
sein.

\begin{Shaded}
\begin{Highlighting}[]
\NormalTok{x =}\StringTok{ }\KeywordTok{c}\NormalTok{(}\DecValTok{1}\OperatorTok{:}\DecValTok{30}\NormalTok{)}
\CommentTok{#typeof() gibt den Datentyp einer Variable wieder }
\KeywordTok{typeof}\NormalTok{(x)}
\end{Highlighting}
\end{Shaded}

\begin{verbatim}
## [1] "integer"
\end{verbatim}

\begin{Shaded}
\begin{Highlighting}[]
\CommentTok{#str() listet den Datentyp und die verschiedenen Ausprägungen einer Variable auf}
\KeywordTok{str}\NormalTok{(x)}
\end{Highlighting}
\end{Shaded}

\begin{verbatim}
##  int [1:30] 1 2 3 4 5 6 7 8 9 10 ...
\end{verbatim}

\begin{Shaded}
\begin{Highlighting}[]
\CommentTok{#length() gibt die Länge einer Liste wieder}
\KeywordTok{length}\NormalTok{(x)}
\end{Highlighting}
\end{Shaded}

\begin{verbatim}
## [1] 30
\end{verbatim}

\section{R-Datentypen}\label{r-datentypen}

R weist die Datentypen automatisch zu - Variablentyp muss nicht explizit
festgelegt werden. Manchmal kann man den Variablentyp ändern, wenn
bestimmte Funktionen durchgeführt werden sollen.

\textbf{integer} sind ganzzahlige Werte im endlichen Bereich
\textbackslash{} - standardisierter Bereich ist normalerweise von -32768
bis +32768 - wenn der Wert größer als 32768 ist, dann gibt es eine Art
Überlauf

\begin{Shaded}
\begin{Highlighting}[]
\NormalTok{x =}\StringTok{ }\KeywordTok{c}\NormalTok{(}\DecValTok{1}\OperatorTok{:}\DecValTok{30}\NormalTok{)}
\KeywordTok{typeof}\NormalTok{(x)}
\end{Highlighting}
\end{Shaded}

\begin{verbatim}
## [1] "integer"
\end{verbatim}

\begin{Shaded}
\begin{Highlighting}[]
\KeywordTok{str}\NormalTok{(x)}
\end{Highlighting}
\end{Shaded}

\begin{verbatim}
##  int [1:30] 1 2 3 4 5 6 7 8 9 10 ...
\end{verbatim}

\textbf{double} bezeichnet Fliesskommazahlen, wird in R auch manchmal
als \textbf{numeric} wiedergegeben.

\begin{Shaded}
\begin{Highlighting}[]
\NormalTok{x =}\StringTok{ }\FloatTok{3.232}
\KeywordTok{str}\NormalTok{(x)}
\end{Highlighting}
\end{Shaded}

\begin{verbatim}
##  num 3.23
\end{verbatim}

\begin{Shaded}
\begin{Highlighting}[]
\KeywordTok{typeof}\NormalTok{(x)}
\end{Highlighting}
\end{Shaded}

\begin{verbatim}
## [1] "double"
\end{verbatim}

\textbf{Character} enthält Buchstabenfolgen. Diese müssen in " " stehen,
um sie von den Variablen zu unterscheiden.

\begin{Shaded}
\begin{Highlighting}[]
\NormalTok{x =}\StringTok{ "Langeweile"}
\NormalTok{x}
\end{Highlighting}
\end{Shaded}

\begin{verbatim}
## [1] "Langeweile"
\end{verbatim}

\begin{Shaded}
\begin{Highlighting}[]
\KeywordTok{str}\NormalTok{(x)}
\end{Highlighting}
\end{Shaded}

\begin{verbatim}
##  chr "Langeweile"
\end{verbatim}

\begin{Shaded}
\begin{Highlighting}[]
\KeywordTok{typeof}\NormalTok{(x)}
\end{Highlighting}
\end{Shaded}

\begin{verbatim}
## [1] "character"
\end{verbatim}

\textbf{Factor} sind Variablen, die bestimmte endliche Werte annehmen
können. Sie bezeichnen kategorische Werte. \textbf{Factor} können als
Zahlen oder Buchstaben dargestellt werden.

\begin{Shaded}
\begin{Highlighting}[]
\NormalTok{x =}\StringTok{ }\KeywordTok{as.factor}\NormalTok{(}\KeywordTok{c}\NormalTok{(}\StringTok{"apfel"}\NormalTok{,}\StringTok{"birne"}\NormalTok{,}\StringTok{"erdbeere"}\NormalTok{,}\StringTok{"erdbeere"}\NormalTok{))}
\NormalTok{y =}\StringTok{ }\KeywordTok{as.factor}\NormalTok{(}\KeywordTok{c}\NormalTok{(}\DecValTok{1}\NormalTok{,}\DecValTok{2}\NormalTok{,}\DecValTok{2}\NormalTok{,}\DecValTok{1}\NormalTok{,}\DecValTok{1}\NormalTok{,}\DecValTok{1}\NormalTok{,}\DecValTok{8}\NormalTok{))}
\KeywordTok{str}\NormalTok{(x)}
\end{Highlighting}
\end{Shaded}

\begin{verbatim}
##  Factor w/ 3 levels "apfel","birne",..: 1 2 3 3
\end{verbatim}

\begin{Shaded}
\begin{Highlighting}[]
\KeywordTok{str}\NormalTok{(y)}
\end{Highlighting}
\end{Shaded}

\begin{verbatim}
##  Factor w/ 3 levels "1","2","8": 1 2 2 1 1 1 3
\end{verbatim}

\textbf{logical} Evaluation einer logischen Frage: Darstellung als
\textbf{TRUE} oder \textbf{FALSE}; kann nur zwei Werte annehmen.

\begin{itemize}
\tightlist
\item
  Siehe hierzu die Inhalte von letzter Woche und die Folien und Übungen
  im
\end{itemize}

\textbf{complex} Daten, die nicht in traditioneller Weise dargestellt
werden.

Die \textbf{str()} Funktion stellt die Datentypen der Werte in einer
Tabelle dar:

\begin{Shaded}
\begin{Highlighting}[]
\CommentTok{# listet Datentypen aller Spalten auf}
\NormalTok{list =}\StringTok{ }\KeywordTok{read.table}\NormalTok{(}\StringTok{"gugVP11.txt"}\NormalTok{,}\DataTypeTok{header =}\NormalTok{ T)}
\KeywordTok{str}\NormalTok{(list)}
\end{Highlighting}
\end{Shaded}

\textbf{str()} listet die Tabelle nach den Spalten auf:

\begin{enumerate}
\def\labelenumi{\arabic{enumi}.}
\tightlist
\item
  zuerst erscheint der Spaltennahme
\item
  dann der Datentyp: \textbf{int} = integer, ganzzahliger Datentyp in
  einem bestimmten Bereich \textbf{Factor}, Daten können nur einen
  bestimmten Wert annehmen
\item
  danach erscheinen die verschiedenen Beobachtungen der jeweiligen
  Spalte
\end{enumerate}

\section{R-Data frames}\label{r-data-frames}

\textbf{Data Frames} ist die Datenstruktur in Form einer Tabelle
(zweidimensional) mit \textbf{Zeilen} und \textbf{Spalten}. Deren Werte
kann eine Mischung aus Zahlen und Buchstaben enthalten.

\begin{Shaded}
\begin{Highlighting}[]
\CommentTok{# ein Data Frame hat folgende Form}
\CommentTok{# diese Information wird später noch einmal wichtig}
\NormalTok{name[}\OperatorTok{<}\NormalTok{zeile}\OperatorTok{>}\NormalTok{, }\OperatorTok{<}\NormalTok{spalte}\OperatorTok{>}\NormalTok{]}
\end{Highlighting}
\end{Shaded}

\begin{Shaded}
\begin{Highlighting}[]
\CommentTok{# so kann man eine Liste generieren}
\CommentTok{# c() steht für "concatenate"", verbinden oder aneinanderreihen }
\NormalTok{y =}\StringTok{ }\KeywordTok{c}\NormalTok{(}\StringTok{"Unsinn"}\NormalTok{,}\StringTok{"Quatsch"}\NormalTok{,}\StringTok{"rot"}\NormalTok{)}
\NormalTok{y}
\end{Highlighting}
\end{Shaded}

\begin{verbatim}
## [1] "Unsinn"  "Quatsch" "rot"
\end{verbatim}

\begin{Shaded}
\begin{Highlighting}[]
\CommentTok{#so kann man einen Data Frame generieren }
\NormalTok{a =}\StringTok{ }\KeywordTok{data.frame}\NormalTok{(}\DecValTok{5}\OperatorTok{:}\DecValTok{7}\NormalTok{,y,}\DecValTok{8}\OperatorTok{:}\DecValTok{10}\NormalTok{, }\DataTypeTok{check.names =} \OtherTok{FALSE}\NormalTok{)}
\NormalTok{a}
\end{Highlighting}
\end{Shaded}

\begin{verbatim}
##   5:7       y 8:10
## 1   5  Unsinn    8
## 2   6 Quatsch    9
## 3   7     rot   10
\end{verbatim}

So sieht der Inhalt des erstellten Data Frame schlußendlich aus; wie
eine Tabelle ohne Überschrift und ohne Spalten und Zeilennamen

\section{R-Matritzen}\label{r-matritzen}

Matritzen in \textbf{R} haben ähnlich wie Dataframes die Form einer
Tabelle, nur bestehen Matritzen nur aus Zahlen und enthalten keine
Buchstaben.

\begin{Shaded}
\begin{Highlighting}[]
\CommentTok{# so kann man eine Matrix von 1 bis 30 aufsteigend erstellen }
\CommentTok{#und diese auf 10 Spalten und 3 Zeilen verteilen}
\NormalTok{a =}\StringTok{ }\KeywordTok{matrix}\NormalTok{(}\DataTypeTok{data =} \DecValTok{1}\OperatorTok{:}\DecValTok{30}\NormalTok{, }\DataTypeTok{ncol =} \DecValTok{10}\NormalTok{, }\DataTypeTok{nrow =} \DecValTok{3}\NormalTok{, }\DataTypeTok{byrow =} \OtherTok{TRUE}\NormalTok{)}

\CommentTok{# so sieht der Inhalt dann aus}
\NormalTok{a}
\end{Highlighting}
\end{Shaded}

\begin{verbatim}
##      [,1] [,2] [,3] [,4] [,5] [,6] [,7] [,8] [,9] [,10]
## [1,]    1    2    3    4    5    6    7    8    9    10
## [2,]   11   12   13   14   15   16   17   18   19    20
## [3,]   21   22   23   24   25   26   27   28   29    30
\end{verbatim}

Mit der Funktion \textbf{dim()} lässt sich die Größe eines Dataframes
oder einer Matritze anzeigen. Nehmen wir mal die selbsterstellte Matrix
mit dem Namen \textbf{a}: \textbf{dim(a)} gibt die Anzahl der Zeilen und
die Anzahl der Spalten der jeweiligen Tabelle wieder.

\begin{Shaded}
\begin{Highlighting}[]
\KeywordTok{dim}\NormalTok{(a)}
\end{Highlighting}
\end{Shaded}

\begin{verbatim}
## [1]  3 10
\end{verbatim}

Hier zeigt sich die schon angedeutete Anordnung in den eckigen Klammern:
zuerst die links die \textbf{Zeile} dann rechts die \textbf{Spalte}.

\begin{Shaded}
\begin{Highlighting}[]
\CommentTok{# folgender Code zeigt, wie man auf einzelne Spalten zugreifen kann}
\CommentTok{# so kann man sich die 1. Spalte der Tabelle anzeigen lassen}
\NormalTok{a[,}\DecValTok{1}\NormalTok{]}
\end{Highlighting}
\end{Shaded}

\begin{verbatim}
## [1]  1 11 21
\end{verbatim}

\begin{Shaded}
\begin{Highlighting}[]
\CommentTok{# hier zeigt sich, wie man sich die 1. Zeile der Tabelle anzeigen lassen kann}
\NormalTok{a[}\DecValTok{1}\NormalTok{,]}
\end{Highlighting}
\end{Shaded}

\begin{verbatim}
##  [1]  1  2  3  4  5  6  7  8  9 10
\end{verbatim}

Die \textbf{mean()} Funktion gibt den Mittelwert von einer Reihe von
Zahlen wieder

\begin{Shaded}
\begin{Highlighting}[]
\CommentTok{# hier wird der Mittelwert der 1. Zeile der Tabelle a ermittelt und ausgegeben}
\KeywordTok{mean}\NormalTok{(a[}\DecValTok{1}\NormalTok{,])}
\end{Highlighting}
\end{Shaded}

\begin{verbatim}
## [1] 5.5
\end{verbatim}

Man kann in der Tabelle den einzelnen Spalten und Zeilen Namen geben.
Falls dies im originalen eingelesenen Datensatz nicht schon geschehen
ist, gibt es hierfür die Funktionen \textbf{colnames()} und
\textbf{rownames()}

\begin{Shaded}
\begin{Highlighting}[]
\CommentTok{# die einzelnen Spalten werden alphabetisch angeordnet}
\KeywordTok{colnames}\NormalTok{(a) =}\StringTok{ }\KeywordTok{c}\NormalTok{(}\StringTok{"a"}\NormalTok{,}\StringTok{"b"}\NormalTok{,}\StringTok{"c"}\NormalTok{,}\StringTok{"d"}\NormalTok{,}\StringTok{"e"}\NormalTok{,}\StringTok{"f"}\NormalTok{,}\StringTok{"g"}\NormalTok{,}\StringTok{"h"}\NormalTok{,}\StringTok{"i"}\NormalTok{,}\StringTok{"j"}\NormalTok{)}

\CommentTok{# die einzelnen Zeilen bekommen so einen eigenen Farbnamen}
\CommentTok{# wenn die Zuweisungen Buchstaben enthalten, müssen diese in Hochkommata geschrieben werden}
\CommentTok{# bei der Zuweisung von Zahlen, müssen diese nicht in Hochkommata stehen}
\KeywordTok{rownames}\NormalTok{(a) =}\StringTok{ }\KeywordTok{c}\NormalTok{(}\StringTok{"rot"}\NormalTok{,}\StringTok{"blau"}\NormalTok{,}\StringTok{"weiss"}\NormalTok{)}

\CommentTok{# gib den Inhalt der Tabelle a aus}
\NormalTok{a}
\end{Highlighting}
\end{Shaded}

\begin{verbatim}
##        a  b  c  d  e  f  g  h  i  j
## rot    1  2  3  4  5  6  7  8  9 10
## blau  11 12 13 14 15 16 17 18 19 20
## weiss 21 22 23 24 25 26 27 28 29 30
\end{verbatim}

\section{R - nützliche Funktionen}\label{r---nuxfctzliche-funktionen}

\textbf{as.factor()} Mit dieser Funktion lässt sich ein Vector als
Faktor kodieren. Ein als Faktor kodierter Vektor kann nur Werte aus
bestimmten vordefiniertem Bereich enthalten.

Kopieren Sie jeweils den Code und lassen Sie sich die Ergebnisse
anzeigen.

\begin{Shaded}
\begin{Highlighting}[]
\NormalTok{x =}\StringTok{ }\KeywordTok{c}\NormalTok{(}\DecValTok{1}\OperatorTok{:}\DecValTok{30}\NormalTok{)}
\KeywordTok{str}\NormalTok{(x)}
\end{Highlighting}
\end{Shaded}

\begin{verbatim}
##  int [1:30] 1 2 3 4 5 6 7 8 9 10 ...
\end{verbatim}

\begin{Shaded}
\begin{Highlighting}[]
\NormalTok{x =}\StringTok{ }\KeywordTok{as.factor}\NormalTok{(x)}
\KeywordTok{str}\NormalTok{(x)}
\end{Highlighting}
\end{Shaded}

\begin{verbatim}
##  Factor w/ 30 levels "1","2","3","4",..: 1 2 3 4 5 6 7 8 9 10 ...
\end{verbatim}

\textbf{levels()} Zeigt die Levels eines Factors an.

\begin{Shaded}
\begin{Highlighting}[]
\NormalTok{x =}\StringTok{ }\KeywordTok{c}\NormalTok{(}\DecValTok{1}\OperatorTok{:}\DecValTok{30}\NormalTok{)}
\NormalTok{x =}\StringTok{ }\KeywordTok{as.factor}\NormalTok{(x)}
\KeywordTok{levels}\NormalTok{(x)}
\end{Highlighting}
\end{Shaded}

\begin{verbatim}
##  [1] "1"  "2"  "3"  "4"  "5"  "6"  "7"  "8"  "9"  "10" "11" "12" "13" "14" "15"
## [16] "16" "17" "18" "19" "20" "21" "22" "23" "24" "25" "26" "27" "28" "29" "30"
\end{verbatim}

\textbf{table()} erstellt eine Übersicht über die Werte einer Spalte.

\begin{Shaded}
\begin{Highlighting}[]
\NormalTok{x =}\StringTok{ }\KeywordTok{c}\NormalTok{(}\DecValTok{1}\OperatorTok{:}\DecValTok{30}\NormalTok{)}
\NormalTok{x =}\StringTok{ }\KeywordTok{as.factor}\NormalTok{(x)}
\KeywordTok{table}\NormalTok{(x)}
\end{Highlighting}
\end{Shaded}

\begin{verbatim}
## x
##  1  2  3  4  5  6  7  8  9 10 11 12 13 14 15 16 17 18 19 20 21 22 23 24 25 26 
##  1  1  1  1  1  1  1  1  1  1  1  1  1  1  1  1  1  1  1  1  1  1  1  1  1  1 
## 27 28 29 30 
##  1  1  1  1
\end{verbatim}

\begin{Shaded}
\begin{Highlighting}[]
\CommentTok{# Hilfefunktionen in R}
\KeywordTok{help}\NormalTok{(}\StringTok{"paketname"}\NormalTok{)}
\NormalTok{?Paketname}
\end{Highlighting}
\end{Shaded}

\begin{Shaded}
\begin{Highlighting}[]
\KeywordTok{install.packages}\NormalTok{(}\StringTok{"dplyr"}\NormalTok{)}
\KeywordTok{library}\NormalTok{(dplyr)}

\CommentTok{# zeige nur Zeilen, die bestimmte Bedingung erfüllen}
\KeywordTok{filter}\NormalTok{()}
\KeywordTok{filter}\NormalTok{(list,time }\OperatorTok{<}\StringTok{ }\DecValTok{150}\NormalTok{)}

\CommentTok{# ordne Zeilen nach einer bestimmten Spalte}
\KeywordTok{arrange}\NormalTok{()}
\KeywordTok{arrange}\NormalTok{(list,time)}
\end{Highlighting}
\end{Shaded}

\begin{Shaded}
\begin{Highlighting}[]
\CommentTok{# zusammenfassen der Daten}
\CommentTok{# min, max, mean, median, var, sd - anwendbar}
\KeywordTok{summarise}\NormalTok{()}
\KeywordTok{summarise}\NormalTok{(list,}\DataTypeTok{avg =} \KeywordTok{mean}\NormalTok{(time))}

\CommentTok{# wähle Spalten nach Namen aus}
\KeywordTok{select}\NormalTok{()}
\KeywordTok{select}\NormalTok{(list, VP, time)}

\CommentTok{# mache Berechnungen und hänge Spalten an}
\KeywordTok{mutate}\NormalTok{()}
\KeywordTok{mutate}\NormalTok{(list, }\DataTypeTok{time2 =}\NormalTok{ time}\OperatorTok{/}\DecValTok{100}\NormalTok{)}
\end{Highlighting}
\end{Shaded}

Eine \textbf{Funktion} ist die Aneinanderreihung auszuführender Befehle.
Man kann sie selbst definieren, schnell wiederholen und schnell
anwenden.

\begin{Shaded}
\begin{Highlighting}[]
\OperatorTok{<}\NormalTok{Name}\OperatorTok{>}\StringTok{ }\ErrorTok{=}\StringTok{ }\ControlFlowTok{function}\NormalTok{(}\OperatorTok{<}\NormalTok{Liste von übergebenen Argumenten}\OperatorTok{>}\NormalTok{)\{}
  \OperatorTok{<}\NormalTok{Befehle, die ausgeführt werden sollen}\OperatorTok{>}
\NormalTok{\}}
\end{Highlighting}
\end{Shaded}

Beispiel einer Funktion. Probiert diese doch einmal aus!

\begin{Shaded}
\begin{Highlighting}[]
\CommentTok{# "percent" berechnet Prozent a von b in Variable d}
\NormalTok{percent =}\StringTok{ }\ControlFlowTok{function}\NormalTok{(a,b)\{ }
\NormalTok{    d =}\StringTok{ }\DecValTok{100}\OperatorTok{/}\NormalTok{b }\OperatorTok{*}\StringTok{ }\NormalTok{a}
\NormalTok{    d}
\NormalTok{\}}
\KeywordTok{percent}\NormalTok{(}\DecValTok{34}\NormalTok{,}\DecValTok{78}\NormalTok{)}
\end{Highlighting}
\end{Shaded}

\begin{verbatim}
## [1] 43.58974
\end{verbatim}

\bibliography{book.bib,packages.bib}

\end{document}
